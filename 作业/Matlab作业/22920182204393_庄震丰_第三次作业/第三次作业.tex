\documentclass[UTF8,a4paper]{article}
\usepackage{fancyhdr}
\usepackage{ctex}
\usepackage{amsmath}
\usepackage{listings}
\usepackage{color}
\lstset{ %
	extendedchars=false,            % Shutdown no-ASCII compatible
	language=Matlab,                % choose the language of the code
	basicstyle=\small\sf,    % the size of the fonts that are used for the code
	tabsize=3,                            % sets default tabsize to 3 spaces
	numbers=left,                   % where to put the line-numbers
	numberstyle=\tiny,              % the size of the fonts that are used for the line-numbers
	stepnumber=1,                   % the step between two line-numbers. If it's 1 each line
	% will be numbered
	numbersep=5pt,                  % how far the line-numbers are from the code   %
	keywordstyle=\color[RGB]{33,33,234},               % keywords
	commentstyle=\color[RGB]{0,0,0},    % comments
	stringstyle=\color[rgb]{0.170,0.187,0.102},      % strings
	backgroundcolor=\color{white}, % choose the background color. You must add \usepackage{color}
	showspaces=false,               % show spaces adding particular underscores
	showstringspaces=false,         % underline spaces within strings
	showtabs=false,                 % show tabs within strings adding particular underscores                frame = single,         % adds a frame around the code
	captionpos=b,                   % sets the caption-position to bottom
	breaklines=true,                % sets automatic line breaking
	breakatwhitespace=false,        % sets if automatic breaks should only happen at whitespace
	title=\lstname,                 % show the filename of files included with \lstinputlisting;
	% also try caption instead of title
	mathescape=true,escapechar=?    % escape to latex with ?..?
	escapeinside={\%*}{*)},         % if you want to add a comment within your code
	%columns=fixed,                  % nice spacing
	%morestring=[m]',                % strings
	%morekeywords={%,...},%          % if you want to add more keywords to the set
	%    break,case,catch,continue,elseif,else,end,for,function,global,%
	%    if,otherwise,persistent,return,switch,try,while,...},%
}
\pagestyle{fancy}
\lhead{}
\chead{}
\rhead{\bfseries The Matlab Homework week 5}
\lfoot{}
\cfoot{\thepage}
\rfoot{}
\renewcommand{\headrulewidth}{0.4pt}
\begin{document}
\begin{center}
    \textbf{\LARGE{Matlab Homework week 5}}\\[0.5cm]
    \normalsize{庄震丰 22920182204393}\\[0.5cm]
    \large{Oct. $19^{th}$, 2019}
\end{center}
\section{Sum}
\subsection{Description}
get the solution and the answer of 
$$
    \sum_{i=0}^{63} 2^i=1+2+2^2+2^3+...+2^63\\
$$
by useing \textit{for} and \textit{while} structure.\\
And give a simple method of the solution.
\subsection{Analysis}
For the second question,
Use the function \textit{sum} in Matlab directly......
\subsection{Codes and Result}
\textbf{Question 1}
\begin{lstlisting}
    s=0;
    for i=0:63
        s=s+2^i;
    end
\end{lstlisting}
\begin{lstlisting}
    s=0;
    i=0;
    while(i<=63)
        s=s+2^i;
        i=i+1;
    end
\end{lstlisting}
\textbf{Question 2}
\begin{lstlisting}
    sum(2.^[0:63]);
\end{lstlisting}
Output:\\
ans=1.844674407370955e+19\\
\section{Funtion of arcsin}
\subsection{Description}
$$
\begin{aligned}
    arcsinx\approx x+ \frac{x^3}{2\cdot 3}+\frac{1 \cdot 3 \cdot x^5}{2\cdot 4 \cdot 5}&+\cdots +\frac{(2n)!x^{2n+1}}{2^{2n}(n!)^2(2n+1)}\\
    \frac{(2n)!x^{2n+1}}{2^{2n}(n!)^2(2n+1)}<&0.02 
\end{aligned}
$$
give the result of approximate value of arcsinx.\\
Hint:\\
use funtion \textit{factorial},the loop structure \textit{while}.
\subsection{Anaylsis}
use factorial in the loop while and add the i by step.
\subsection{Code and Result}
\begin{lstlisting}
    function r=arcsin(x)
    n=0;
    r=0;
    while(factorial(2*n)*x^(2*n+1)/(2^(2*n)*(factorial(n)^2)*(2*n+1))>=0.002)
        r=r+factorial(2*n)*x^(2*n+1)/(2^(2*n)*(factorial(n)^2)*(2*n+1));
        n=n+1;
    end
    end    
\end{lstlisting}
>>arcsin(0.5)
Output:\\
    ans=0.5232\\
($pi/6\approx 0.5236$)
\section{Solving gcd and lcm and judge prime number}
\subsection{Description}

\noindent Like title.\\
For the second Question, 1 means the number is prime, else 0.
\subsection{Anaylsis}
\textbf{Question 1} \\
Euclidean algorithm.\\
\textbf{Question 2}\\
Each i in loop [2,$\sqrt{n}$] can't be divisible,otherwise the number is not prime number.\\
use the function \textit{mod} and \textit{floor}.
\subsection{Code and Result}
\noindent Question1:\\
\begin{lstlisting}
    function [b,y]=by(m,n)
    m0=m;n0=n;
    z=mod(m0,n0);
    while(z~=0)
        m0=n0;n0=z;
        z=mod(m0,n0);
    end
    b=n0;y=m*n/b; %gcd b,lcm y
    end
\end{lstlisting}
>>[b,y]=by[9,15]\\
Output:\\
b=3,y=45\\\\
Question2:\\
\begin{lstlisting}
    function judge=sushu(n)
    judge=1;
    for i=2:floor(sqrt(n))+1
         if (mod(n,i)==0) judge=0;
         end
    end
    end
\end{lstlisting}
>>sushu(6)\\
Output:\\
ans=0\\\\
>>sushu(11)\\
Output:\\
ans=1\\
\section{Magic matrix}
\subsection{Description}
In MATLAB, the magic() function is called the cube matrix function, which automatically generates a special N-order square matrix (where N=1, 2,3, 4, 5.....). These N-order squares have a common characteristic that the sum of the elements in each row, column or diagonal is equal and constant. Try to design a function mag(n) to verify its wonderful properties for the N-order cube.
\subsection{Anaylsis}
\noindent use function \textit{diag} to get the vetctor of diagonal elements.\\
use function \textit{sum} to summary the each row and column and diagonal,and judge whether they are equal.
\subsection{Code and Result}
\begin{lstlisting}
function [judge]=mag(M)
    n=length(M);
    judge=(sum(find(sum(M,2)~=(1+n^2)*n/2))==0)&(sum(find(sum(M,1)~=(1+n^2)*n/2))==0)&(sum(diag(M))==(1+n^2)*n/2)&(sum(diag(rot90(M)))==(1+n^2)*n/2);
    % 1:is magic matrix
    % 0:not magic matrix
end  
\end{lstlisting}
>>mag(magic(5))\\
Output:\\
ans=1
\section{Filter}
\subsection{Description}
\noindent Find the number of [2,999] that satisfies the following conditions at the same time\\
(1) The sum of the numbers of the numbers is an odd number\\
(2) The number is prime
\subsection{Anaylsis}
\noindent use function \textit{sushu} to get prime numbers.\\
use the \textit{sum(sum(num2str(i)-'0')} to solve the value of numbers of the numbers.
\subsection{Code and Result}
\begin{lstlisting}
for i=2:999
    if (mod(sum(num2str(i)-'0'),2)==1&&sushu(i)==1)
        disp(i);
    end
end
\end{lstlisting}
Output:\\
ans=
[3,5,7,23,29,41,43,47,61,67,83,89,113,131,137,139,151,157,173,179,191,193,197,199,223,227,229,241,263,
269,281,283,311,313,317,331,337,353,359,373,379,397,401,409,421,443,449,461,463,467,487,557,571,577,593,
599,601,607,641,643,647,661,683,719,733,739,751,757,773,797,809,821,823,827,829,863,881,883,887,911,919,
937,953,971,977,991,997]
\end{document}