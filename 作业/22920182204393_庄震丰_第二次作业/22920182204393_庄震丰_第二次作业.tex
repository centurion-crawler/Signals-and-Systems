\documentclass[UTF8,a4paper]{article}
\usepackage{fancyhdr}
\usepackage{ctex}
\usepackage{amsmath}
\usepackage{listings}
\usepackage{color}
\lstset{ %
	extendedchars=false,            % Shutdown no-ASCII compatible
	language=Matlab,                % choose the language of the code
	basicstyle=\small\sf,    % the size of the fonts that are used for the code
	tabsize=3,                            % sets default tabsize to 3 spaces
	numbers=left,                   % where to put the line-numbers
	numberstyle=\tiny,              % the size of the fonts that are used for the line-numbers
	stepnumber=1,                   % the step between two line-numbers. If it's 1 each line
	% will be numbered
	numbersep=5pt,                  % how far the line-numbers are from the code   %
	keywordstyle=\color[RGB]{33,33,234},               % keywords
	commentstyle=\color[RGB]{0,0,0},    % comments
	stringstyle=\color[rgb]{0.170,0.187,0.102},      % strings
	backgroundcolor=\color{white}, % choose the background color. You must add \usepackage{color}
	showspaces=false,               % show spaces adding particular underscores
	showstringspaces=false,         % underline spaces within strings
	showtabs=false,                 % show tabs within strings adding particular underscores                frame = single,         % adds a frame around the code
	captionpos=b,                   % sets the caption-position to bottom
	breaklines=true,                % sets automatic line breaking
	breakatwhitespace=false,        % sets if automatic breaks should only happen at whitespace
	title=\lstname,                 % show the filename of files included with \lstinputlisting;
	% also try caption instead of title
	mathescape=true,escapechar=?    % escape to latex with ?..?
	escapeinside={\%*}{*)},         % if you want to add a comment within your code
	%columns=fixed,                  % nice spacing
	%morestring=[m]',                % strings
	%morekeywords={%,...},%          % if you want to add more keywords to the set
	%    break,case,catch,continue,elseif,else,end,for,function,global,%
	%    if,otherwise,persistent,return,switch,try,while,...},%
}
\pagestyle{fancy}
\lhead{}
\chead{}
\rhead{\bfseries The Matlab Homework week 2}
\lfoot{}
\cfoot{\thepage}
\rfoot{}
\renewcommand{\headrulewidth}{0.4pt}
\begin{document}
\begin{center}
    \textbf{\LARGE{Matlab Homework week 2}}\\[0.5cm]
    \normalsize{庄震丰 22920182204393}\\[0.5cm]
    \large{Sept. $26^{th}$, 2019}
\end{center}
\section{Martix Multiplication}
\subsection{Description}
\[A=\begin{bmatrix}
     1 & 2 &  3\\
    -2 & 0 &  0\\
     1 & 0 &  1\\
    -1 & 2 & -3
\end{bmatrix}, 
B=\begin{bmatrix}
    -1 & 3 \\
    -2 & 2 \\
     2 & 1
\end{bmatrix}\]
solve the matrix $C=A\times B$
\subsection{Analysis}
Use the operator $*$ in Matlab directly......
\subsection{Codes and Result}
\begin{lstlisting}
 A=[1,2,3;-2,0,0;1,0,1;-1,2,-3];
 B=[-1,3;-2,2;2,1];
 A*B;
\end{lstlisting}
Result:\\
\[ans=\begin{bmatrix}
    1 & 10\\
    2 & -6\\
    1 & 4\\
    -9 & -2
\end{bmatrix}\]
\\
\section{Solving linear equations}
\subsection{Description}
\begin{equation}
    \left\{
    \begin{aligned} % \begin{eqnarray}好像也可以。
    2x_1-x_2+3x_3&=5\\
    3x_1+x_2-5x_3&=5\\
    4x_1-x_2+x_3&=9
    \end{aligned}
    \right.
\end{equation}
Solve the unknown variables $x_1,x_2,x_3$ 
\subsection{Anaylsis}
Use coefficient matrix A , vector \textbf{b} repersents the numbers on the right side of equal sign,
$$
\begin{aligned}
     A\cdot \textbf{x}=\textbf{b}&\\
     \textbf{x}=A^{-1}\cdot & \textbf{b}
\end{aligned}
$$
and the function \textbf{\textit{inv}} in Matlab maybe useful for the solution to inverse matrix.
\subsection{Code and Result}
\begin{lstlisting}
    A=[2,-1,3;3,1,-5;4,-1,1];
    b=[5;5;9];
    x=inv(A)*b
\end{lstlisting}
Result:\\
\[\textbf{x}=\begin{bmatrix}
    2\\
    -1\\
    0
\end{bmatrix}\]
\\
\section{Solving roots of the equation}
\subsection{Description}
There is a expression
$$
(x-4)(x+5)(x^2-6x+9)
$$
Expand it into a polynomial form and find the root of 
its corresponding unary n-order equation
\subsection{Anaylsis}
Use vectors to repersent expressions\\
If
$$
\begin{aligned}
A(s)=a_{n}s^{n}+a_{n-1}s^{n-1} \dots a_1 s+a_0\\
\end{aligned}
$$
let
\[P=\begin{bmatrix}
    a_n,a_{n-1},a_{n-2},\dots,a_1,a_0  
\end{bmatrix} \] in matlab language.
We also find the function \textbf{\textit{conv}} can make Multiplication between matrices in Matlab.\\
\indent while the function \textbf{\textit{roots}} can give the solution of the equation.
\subsection{Code and Result}
\begin{lstlisting}
    A=[1,-4];
    B=[1,5];
    C=[1,-6,9];
    D=conv(conv(A,B),C)
    x=roots(D)
\end{lstlisting}
Result:\\
$x_1=-5,x_2=x_3=3,x_4=4$
\section{Polynomial operation}
\subsection{Description}
There is a expression
$$
3x^6+12x^5+4x^4+7x^3+8x+1
$$
Give the Result of that divided by $(x-3)(x^3+5x)$.
\subsection{Anaylsis}
Use vectors to repersent expressions(the same solution way in problem 3).
We also find the function \textbf{[R,v]deconv(A,B)} can make Polynomial vector \textbf{A} divided by Polynomial vector \textbf{B}.\\
which means,$A=R\times B+v$.
\subsection{Code and Result}
\begin{lstlisting}
    A=[3 12 4 7 0 8 1];
    B=conv([1,-3],[1,0,5,0]);
    [R v]=deconv(A,B)    
\end{lstlisting}
Result:\\
\[R=\begin{bmatrix}
    3,21,52  
\end{bmatrix} 
,v=\begin{bmatrix}
    0,0,0,103,55,788,1  
\end{bmatrix} 
\]\\
$R=3x^2+21x+52\\
v=103x^3+55x^2+788x+1$
\end{document}